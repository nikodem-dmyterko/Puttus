%%
%  ******************************************************************************
%  * #file    Szablon_raportu_SM_Latex.tex
%  * #author  Adrian Wójcik   adrian.wojcik(at)put.poznan.pl
%  *          
%  * #version 2.0
%  * #date    09 Mar 2021
%  * #brief   Szablon raportu laboratoryjnego : Systemy mikroprocesorowe
%  *
%  ******************************************************************************
%% 
\documentclass[11pt, a4paper]{article}

\usepackage[pl]{Szablon_raportu_SM_Latex} % [pl] or [en]

%%% Laboratory info %%%
\university{Politechnika Poznańska}
\faculty{Wydział Automatyki, Robotyki i Elektrotechniki}
\institute{Instytut Robotyki i Inteligencji Maszynowej}
\department{Zakład Sterowania i Elektroniki Przemysłowej}
\lab{Systemy mikroprocesorowe}
\instructor{mgr inż. Adrian Wójcik}
\instructoremail{Adrian.Wojcik@put.poznan.pl}
\comment{Raport laboratoryjny}

%%% Report info %%%%

% Laboratory exercise:
\maintitle{ \guillemotleft Ćwiczenie laboratoryjne\guillemotright }
\shorttitle{ \guillemotleft Ćwiczenie laboratoryjne\guillemotright }

% #1 author info
\firstauthor{ \guillemotleft Imię\#1 Nazwisko\#1, Nr indeksu\#1\guillemotright }
\firstauthoremail{ \guillemotleft Imie01.Nazwisko02@put.poznan.pl\guillemotright }

% #2 author info (leave blank if only one author)
\secondauthor{ \guillemotleft Imię\#2 Nazwisko\#2, Nr indeksu\#2\guillemotright }
\secondauthoremail{ \guillemotleft Imie02.Nazwisko02@put.poznan.pl\guillemotright }

% Report submittion 
\date{\guillemotleft DD-MM-YYYY\guillemotright} 

\addbibresource{Szablon_raportu_SM_Latex.bib}

%%
%%

\begin{document}

%% First page %%
\mainpage
\newpage

%% Table of contents
\tableofcontents
\thispagestyle{fancy}
\newpage

\section*{Wstęp} \addcontentsline{toc}{section}{Wstęp}
We wstępie do raportu laboratoryjnego należy opisać elementy wspólne dla wykonywanych zadań, np.: sprzęt i narzędzia programistyczne użyte do ćwiczeń. Do wprowadzenia można również dołączyć obrazy, tabele, zestawienia lub wzory odnoszące się do całego raportu.

\section{Zadanie \#1}
Każde z zadań laboratoryjnych zawartych w raporcie powinno być opisane w czterech etapach: \textbf{specyfikacja} zadania, prezentacja \textbf{implementacji} rozwiązania, prezentacja \textbf{wyników testu} (symulacji lub eksperymentu) oraz \textbf{wniosek} wyciągnięty na podstawie poprzednich etapów.

\subsection{Specyfikacja}
Specyfikacja techniczna zadania powinna zwięźle, ale jasno opisywać wymagania, które ma spełnić układ lub program wykonujący zadanie oraz sposób weryfikacji czy wymagania zostały spełnione.

\subsection{Implementacja}
Prezentacja implementacji powinna zawierać krótki opis rozwiązania zadania i jego dokumentację. W zależności od treści zadania dokumentacja implementacji może zawierać listingi (np. Listing \ref{lst:lst1}), tabele (np. Tab. \ref{tab:tab1}), rysunki (np. schematy lub fotografie układu) i formuły. Pamiętaj o wymaganiach redakcyjnych.

%% LISTING EXAMPLE BEGIN
\lstinputlisting[label={lst:lst1}, style=lstC,
caption={Pętla główna programu: programowa detekcja zbocza na wejściu cyfrowym.}]
{lst/lst1.c} 

\vspace{0.5cm}
%% LISTING EXAMPLE END

%% TABLE EXAMPLE BEGIN
\begin{table}[H]
\caption{Połącznie cyfrowego czujnika natężenia światła BH1750 do zestawu NUCLEO-F746ZG za pomocą magistrali I2C \cite{bh1750}}
\label{tab:tab1}
\centering
\begin{tabularx}{10cm}{ | >{\hsize=1.5cm}C | C | >{\hsize=1.5cm}C | C |}
  % table header
  \hline
  \multicolumn{2}{|c|}{NUCLEO-F746ZG} & \multicolumn{2}{c|}{BH1750} \\ \hline
  Pin \# & Pin name & Pin \# & Pin name \\ \hline
  % table content
  - & 3V3     & 1 & VCC \\
  - & GND      & 2 & GND \\
  D15 & I2C1 SCL & 3 & SCL \\
  D14 & I2C1 SDA & 4 & SDA \\
  - & GND      & 5 & ADDR \\
  \hline
\end{tabularx}
\end{table}
%% TABLE EXAMPLE END


\subsection{Wyniki testów}
Prezentacja wyników testów powinna polegać na udokumentowaniu pracy testowanej aplikacji za pomocą debuggera wchodzącego w skład środowiska STM32CubeIDE \cite{ide} oraz w razie potrzeby uzupełniona o udokumentowanie pracy zewnętrznej aplikacji komunikującej się z testowaną aplikacją wbudowaną lub pracy testowanej aplikacji na platformie sprzętowej, np. z użyciem oscyloskopu. \\

Większość zadań wymagać może prezentacji stanu elementów interfejsu debuggera. Proszę pamiętać, że narzędzia systemowe (np. Narzędzie Wycinanie w systemie Windows) umożliwią wybór poszczególnych okien interfejsu – przykład na Rys. \ref{fig:fig2}. Pamiętaj o wymaganiach redakcyjnych.

%% FIGURE EXAMPLE BEGIN
\begin{figure}[H]
	\includecgraphics{fig1}
	\caption{Przykładowy wycinek zrzutu ekranu z elementem interfejsu debuggera}
	\label{fig:fig1}
\end{figure}
%% FIGURE EXAMPLE BEGIN

Podstawowym narzędziem pozwalającym na wizualizację wyników testów jest narzędzie \textit{Serial wire viewer}. Narzędzie to, poza możliwością podglądu zmiennych w programie w czasie rzeczywistym. Na Rys. \ref{fig:fig2}. przedstawiono przykładowy wykres przebiegu czasowego.

%% FIGURE EXAMPLE BEGIN
\begin{figure}[H]
	\includecgraphics[width=12cm]{fig2}
	\caption{Przykładowy wykres uzyskany za pomocą narzędzia SWV Data Trace Timeline Graph.}
	\label{fig:fig2}
\end{figure}
%% FIGURE EXAMPLE BEGIN

\subsection{Wnioski}
Podsumowanie powinno zawierać wnioski dotyczące tego, czy na podstawie \textbf{testów} udało się stwierdzić czy zaprezentowana \textbf{implementacja} rozwiązania spełnia założoną \textbf{specyfikację}. Jeżeli zadanie udało się rozwiązać tylko częściowo należy podać przyczyny takiego stanu rzeczy.

\newpage

\section{Zadanie \#2}

\subsection{Specyfikacja}
\guillemotleft TODO \guillemotright

\subsection{Implementacja}
\guillemotleft TODO \guillemotright

\subsection{Wyniki testów}
\guillemotleft TODO \guillemotright

\subsection{Wnioski}
\guillemotleft TODO \guillemotright

\newpage

\section*{Podsumowanie} \addcontentsline{toc}{section}{Podsumowanie}
Podsumowanie raportu powinno wskazywać, ile i które zadania zostały rozwiązane poprawnie, a które częściowo lub wcale. Opis wszelkich problemów technicznych napotkanych podczas pracy. Możesz opisać, co okazało się trudne w zadaniu.

\printbibliography[heading=bibintoc]

\end{document}